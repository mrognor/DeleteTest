If you have worked with sockets directly before, then you know that there are blocking and non-\/blocking sockets. They differ in how messages are received in user programs. The blocking socket stops the thread in which it is working and waits for information to be received, execution will continue only after that. A non-\/blocking socket does not stop the thread. You don\textquotesingle{}t work with sockets in the library, but you can create an abstract analogue of a non-\/blocking socket. ~\newline


Non-\/blocking mode should work like this. You have a kind of loop in which the logic of your application works. Inside this loop, you check for a new message from the server and if there is one, then you read it and execute certain logic ~\newline


And this is how you can do it ~\newline


First of all, you should save the last incoming message to some variable. Also, for convenience, you can have a Boolean variable to store information about whether this message was read earlier or it is new ~\newline
 \begin{DoxyVerb}void ServerMessageHandler(std::string message)
{
    LastMessage = message;
    isNewMessage = true;
}
\end{DoxyVerb}


Secondly, you must implement a function to receive the last message and to receive information about the arrival of a new message ~\newline
 \begin{DoxyVerb}public:
    bool IsNewMessage()
    {
        return isNewMessage;
    }

    std::string GetLastMessage()
    {
        isNewMessage = false;
        return LastMessage;
    }
\end{DoxyVerb}


Third in the main application loop you have to check if a new message has arrived and process it ~\newline
 \begin{DoxyVerb}while (window.isOpen())
{
    if (Client.IsNewMessage())
    {
        DoSomeLogic();
    }
}
\end{DoxyVerb}
 